\documentclass[a4paper, 14pt]{extarticle}

% Поля
%--------------------------------------
\usepackage{geometry}
\geometry{a4paper,tmargin=2cm,bmargin=2cm,lmargin=3cm,rmargin=1cm}
%--------------------------------------


%Russian-specific packages
%--------------------------------------
\usepackage[T2A]{fontenc}
\usepackage[utf8]{inputenc} 
\usepackage[english, main=russian]{babel}
%--------------------------------------

\usepackage{textcomp}

% Красная строка
%--------------------------------------
\usepackage{indentfirst}               
%--------------------------------------             


%Graphics
%--------------------------------------
\usepackage{graphicx}
\graphicspath{ {./images/} }
\usepackage{wrapfig}
%--------------------------------------

% Полуторный интервал
%--------------------------------------
\linespread{1.3}                    
%--------------------------------------

%Выравнивание и переносы
%--------------------------------------
% Избавляемся от переполнений
\sloppy
% Запрещаем разрыв страницы после первой строки абзаца
\clubpenalty=10000
% Запрещаем разрыв страницы после последней строки абзаца
\widowpenalty=10000
%--------------------------------------

%Списки
\usepackage{enumitem}

%Подписи
\usepackage{caption} 

\newenvironment{longlisting}{\captionsetup{type=listing}}{}

%Гиперссылки
\usepackage{hyperref}

\hypersetup{
  colorlinks=true,
  unicode=true,
}

%Рисунки
%--------------------------------------
\DeclareCaptionLabelSeparator*{emdash}{~--- }
\captionsetup[figure]{labelsep=emdash,font=onehalfspacing,position=bottom}
%--------------------------------------

\usepackage{tempora}

%Листинги
%--------------------------------------
\usepackage{minted}

\renewcommand\listingscaption{Листинг}
\setminted[cpp]{
  frame=single,
  fontsize=\small,
  linenos,
  xleftmargin=1.5em,
}
%--------------------------------------

%%% Математические пакеты %%%
%--------------------------------------
\usepackage{amsthm,amsfonts,amsmath,amssymb,amscd}  % Математические дополнения от AMS
\usepackage{mathtools}                              % Добавляет окружение multlined
\usepackage[perpage]{footmisc}
%--------------------------------------

%--------------------------------------
%			НАЧАЛО ДОКУМЕНТА
%--------------------------------------

\begin{document}

%--------------------------------------
%			ТИТУЛЬНЫЙ ЛИСТ
%--------------------------------------
\begin{titlepage}
\thispagestyle{empty}
\newpage


%Шапка титульного листа
%--------------------------------------
\vspace*{-60pt}
\hspace{-65pt}
\begin{minipage}{0.3\textwidth}
\hspace*{-20pt}\centering
\includegraphics[width=\textwidth]{emblem}
\end{minipage}
\begin{minipage}{0.67\textwidth}\small \textbf{
\vspace*{-0.7ex}
\hspace*{-6pt}\centerline{Министерство науки и высшего образования Российской Федерации}
\vspace*{-0.7ex}
\centerline{Федеральное государственное бюджетное образовательное учреждение }
\vspace*{-0.7ex}
\centerline{высшего образования}
\vspace*{-0.7ex}
\centerline{<<Московский государственный технический университет}
\vspace*{-0.7ex}
\centerline{имени Н.Э. Баумана}
\vspace*{-0.7ex}
\centerline{(национальный исследовательский университет)>>}
\vspace*{-0.7ex}
\centerline{(МГТУ им. Н.Э. Баумана)}}
\end{minipage}
%--------------------------------------

%Полосы
%--------------------------------------
\vspace{-25pt}
\hspace{-35pt}\rule{\textwidth}{2.3pt}

\vspace*{-20.3pt}
\hspace{-35pt}\rule{\textwidth}{0.4pt}
%--------------------------------------

\vspace{1.5ex}
\hspace{-35pt} \noindent \small ФАКУЛЬТЕТ\hspace{80pt} <<Информатика и системы управления>>

\vspace*{-16pt}
\hspace{47pt}\rule{0.83\textwidth}{0.4pt}

\vspace{0.5ex}
\hspace{-35pt} \noindent \small КАФЕДРА\hspace{50pt} <<Теоретическая информатика и компьютерные технологии>>

\vspace*{-16pt}
\hspace{30pt}\rule{0.866\textwidth}{0.4pt}
  
\vspace{11em}

\begin{center}
\Large {\bf Домашняя работа №~3} \\ 
\large {\bf по курсу <<Теория искусственных нейронных сетей>>} \\
\large <<Методы многомерного поиска>> 
\end{center}\normalsize

\vspace{8em}


\begin{flushright}
  {Студент группы ИУ9-71Б Афанасьев И. \hspace*{15pt}\\ 
  \vspace{2ex}
  Преподаватель Каганов Ю.Т.\hspace*{15pt}}
\end{flushright}

\bigskip

\vfill
 

\begin{center}
\textsl{Москва 2024}
\end{center}
\end{titlepage}
%--------------------------------------
%		КОНЕЦ ТИТУЛЬНОГО ЛИСТА
%--------------------------------------

\renewcommand{\ttdefault}{pcr}

\setlength{\tabcolsep}{3pt}
\newpage
\setcounter{page}{2}

\section{Цель работы}

\begin{enumerate}
  \item Изучение алгоритмов многомерного поиска 1-го и 2-го порядка.
  \item Разработка программ реализации алгоритмов многомерного поиска 1-го и 2-го порядка.
  \item Вычисление экстремумов функции.
\end{enumerate}

\section{Постановка задачи}

Требуется найти минимум тестовой функции Розенброка 
\begin{displaymath}
  f(x) = \sum_{i=1}^{n-1}[a (x_i^2 - x_{i+1})^2 + b (x_i - 1)^2] + f_0
\end{displaymath}
методами сопряжённых градиентов (Флетчера-Ривза и Полака-Рибьера), квазиньютоновским методом (Девидона-Флетчера-Пауэлла),
методом Левенберга-Марквардта.

Вариант №~4: $a = 250$, $b = 2$, $f_0 = 50$, $n = 2$.

\section{Реализация}

Программа написана на языке C++ с использованием библиотеки \href{https://eigen.tuxfamily.org/index.php?title=Main_Page}{Eigen}.
В листинге \ref{lst:main.cc} приводится исходный код программы.
 
\begin{longlisting}
  \caption{Файл \texttt{main.cc}}
  \inputminted{cpp}{../src/main.cc}
  \label{lst:main.cc}
\end{longlisting}

\section{Результаты работы методов}

Рассматривается функция $f(x) = 250 (x_1^2 - x_2)^2 + 2 (x_1 - 1)^2 + 50$. Глобальный минимум функции $f(x)$ достигается в точке
$(1, 1)$, и равен $50$. 

Стартовой точкой методов многомерного поиска выбрана точка $(100, 100)$.
Построение последовательности $\{x^k\}$, $k = 0, 1, \ldots$, заканчивается в точке $x^k$, если $||\nabla f(x^k)|| \le \varepsilon_1$, где
$\varepsilon_1 = 10^{-9}$, или при одновременном выполнении неравенств $||x^{k+1} - x^k|| < \delta$, $|f(x^{k+1}) - f(x^k)| < \varepsilon_2$,
где $\delta = 10^{-9}$ и $\varepsilon_2 = 10^{-9}$, или если достигается предельное число итераций $100$.

В качестве метода одномерного поиска используется поиск Фибоначчи на отрезке $[-10, 10]$.

\subsection{Метод наискорейшего градиентного спуска}

Метод завершает работу при достижении предельного числа итераций в точке $x \approx (10.029, 100.534)$, $f(x) \approx 213.065$.
Результаты последних 10 итераций:
\begin{small}
\begin{Verbatim}
  Iteration 89, x = (10.03148807122927, 100.63172017630485), f(x) = 213.1357874561481
  Iteration 90, x = (-10.013013775454562, 100.26484032873051), f(x) = 292.5777748566868
  Iteration 91, x = (-10.012813057598942, 100.25387819301962), f(x) = 292.56572485538834
  Iteration 92, x = (10.030816294660914, 100.62088009295822), f(x) = 213.1145341013168
  Iteration 93, x = (10.03085179084519, 100.61896304568614), f(x) = 213.11280598565742
  Iteration 94, x = (-10.012165216528189, 100.24784778703183), f(x) = 292.54039553811526
  Iteration 95, x = (-10.011961170187258, 100.23678868964839), f(x) = 292.5282388696106
  Iteration 96, x = (10.030103267247357, 100.6065760093384), f(x) = 213.08877806311668
  Iteration 97, x = (10.030138416760341, 100.60463698759254), f(x) = 213.08703020927487
  Iteration 98, x = (-10.011839631207621, 100.2413283517329), f(x) = 292.526054343388
  Iteration 99, x = (-10.011643535379804, 100.23050248101643), f(x) = 292.5141539522706
\end{Verbatim}
\end{small}

\subsection{Метод Флетчера-Ривза}

Метод завершает работу при достижении предельного числа итераций в точке $x \approx (-0.926, 0.864)$, $f(x) \approx 57.428$.
Результаты последних 10 итераций:
\begin{small}
\begin{Verbatim}
Iteration 89, x = (-4.08412185948198, 16.68654607052422), f(x) = 101.70713547026128
Iteration 90, x = (3.7871653724492456, 14.345454132843685), f(x) = 65.53858749642693
Iteration 91, x = (3.770473569803468, 14.198729558724722), f(x) = 65.42973675955056
Iteration 92, x = (-0.1589940079574168, 0.027641936171113102), f(x) = 52.68792997607468
Iteration 93, x = (0.02932280121196934, -0.022889320348931187), f(x) = 52.025433944536545
Iteration 94, x = (0.0120311161498119, -0.010681364705851234), f(x) = 51.98146620867051
Iteration 95, x = (0.05106332672405491, 0.01578724182183233), f(x) = 51.8443882600091
Iteration 96, x = (0.050137085153883425, 0.003071311900725104), f(x) = 51.80455683914451
Iteration 97, x = (0.42951984809284094, 0.17336087010422682), f(x) = 50.68184456747179
Iteration 98, x = (-0.9281914600700055, 0.8730420131046611), f(x) = 57.46892221780569
Iteration 99, x = (-0.9283318868338613, 0.8619916121493209), f(x) = 57.4369369015416
\end{Verbatim}
\end{small}

\subsection{Метод Полака-Рибьера}

Метод завершает работу при достижении предельного числа итераций в точке $x \approx (1.005, 1.009)$, $f(x) \approx 50$.
Результаты последних 10 итераций:
\begin{small}
\begin{Verbatim}
Iteration 89, x = (1.005633730653852, 1.011307936620449), f(x) = 50.000063496923296
Iteration 90, x = (1.0056302314589185, 1.0113098232318996), f(x) = 50.00006347698859
Iteration 91, x = (1.0043009072222773, 1.0088441122433192), f(x) = 50.0000495172154
Iteration 92, x = (1.0046386606693454, 1.009411457259613), f(x) = 50.000046205091216
Iteration 93, x = (1.0046763847195765, 1.009388999323969), f(x) = 50.0000437887099
Iteration 94, x = (1.0045945627761548, 1.009272728661987), f(x) = 50.000043196361176
Iteration 95, x = (1.0046125853667958, 1.009260045808578), f(x) = 50.000042598121624
Iteration 96, x = (1.0045875577058334, 1.0092340216735158), f(x) = 50.00004244972722
Iteration 97, x = (1.0045961353827297, 1.0092257724546962), f(x) = 50.00004228721986
Iteration 98, x = (1.0045874347221277, 1.0092210134289983), f(x) = 50.00004224660997
Iteration 99, x = (1.004591481057644, 1.009213616254047), f(x) = 50.00004218630451
\end{Verbatim}
\end{small}

\subsection{Метод Девидона-Флетчера-Пауэлла}

Метод завершает работу при достижении предельного числа итераций в точке $x \approx (1, 1)$, $f(x) = 50$.
Результаты последних 10 итераций:
\begin{small}
\begin{Verbatim}
Iteration 89, x = (0.9999999867851308, 0.9999999735167926), f(x) = 50
Iteration 90, x = (1.0000000041995718, 1.0000000084161353), f(x) = 50
Iteration 91, x = (0.9999999957367549, 0.9999999914562606), f(x) = 50
Iteration 92, x = (0.9999999772863342, 0.9999999544807684), f(x) = 50
Iteration 93, x = (1.0000000072181758, 1.0000000144655563), f(x) = 50
Iteration 94, x = (0.9999999983678916, 0.9999999967291797), f(x) = 50
Iteration 95, x = (0.9999999687412185, 0.9999999373559662), f(x) = 50
Iteration 96, x = (0.9999999992217837, 0.9999999984404188), f(x) = 50
Iteration 97, x = (0.9999999786008551, 0.9999999571151265), f(x) = 50
Iteration 98, x = (1.0000000068004409, 1.0000000136283973), f(x) = 50
Iteration 99, x = (1.0000000217681058, 1.0000000436242877), f(x) = 50
\end{Verbatim}
\end{small}

\subsection{Метод Левенберга-Марквардта}

Метод завершает работу в точке $x \approx (1, 1)$, $f(x) = 50$ с выполнением условия $||\nabla f(x)|| < \varepsilon_1$.
Результаты последних 10 итераций:
\begin{small}
\begin{Verbatim}
Iteration 25, x = (19.171649934879973, 367.55120699842826), f(x) = 710.4179503490125
Iteration 26, x = (8.335900, -47.925231), f(x) = 3446579.7627404225
Iteration 27, x = (8.335611004239171, 69.48185088949523), f(x) = 157.62245598989048
Iteration 28, x = (2.3380938952157795, -30.503290027030715), f(x) = 323513.32201214595
Iteration 29, x = (2.33793396318421, 5.465763676818462), f(x) = 53.58014193612435
Iteration 30, x = (1.0633452377864907, -0.4938591153397933), f(x) = 709.8086188294966
Iteration 31, x = (1.0631845954627233, 1.1303537133729828), f(x) = 50.00798460130335
Iteration 32, x = (1.0002259021977198, 0.9964883775294373), f(x) = 50.003927391325206
Iteration 33, x = (1.0001118823222457, 1.0002237597086907), f(x) = 50.00000002503538
Iteration 34, x = (1.000000042156586, 1.0000000719382793), f(x) = 50.00000000000004
Iteration 35, x = (1.0000000000070761, 1.0000000000141718), f(x) = 50
\end{Verbatim}
\end{small}

\section{Вывод}

Метод наискорейшего градиентного спуска не обнаруживает точку глобального минимума функции $f(x)$
(идёт застревание между <<оврагами>> функции); остальные методы достаточно близко обнаруживают эту точку. Среди методов сопряжённых
градиентов лучший результат показывает метод Полака-Рибьера. Квазиньютоновский метод и метод Левенберга-Марквардта обнаруживают минимум
со сколь угодно большой точностью. Метод Левенберга-Марквардта справляется с задачей за наименьшее число итераций, однако в общем случае
требует вычисление обратной матрицы Гессе, что занимает больше времени.

\end{document}
