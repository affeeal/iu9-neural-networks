\documentclass[a4paper, 14pt]{extarticle}

% Поля
%--------------------------------------
\usepackage{geometry}
\geometry{a4paper,tmargin=2cm,bmargin=2cm,lmargin=3cm,rmargin=1cm}
%--------------------------------------


%Russian-specific packages
%--------------------------------------
\usepackage[T2A]{fontenc}
\usepackage[utf8]{inputenc} 
\usepackage[english, main=russian]{babel}
%--------------------------------------

\usepackage{textcomp}

% Красная строка
%--------------------------------------
\usepackage{indentfirst}               
%--------------------------------------             


%Graphics
%--------------------------------------
\usepackage{graphicx}
\graphicspath{ {./images/} }
\usepackage{wrapfig}
%--------------------------------------

% Полуторный интервал
%--------------------------------------
\linespread{1.3}                    
%--------------------------------------

%Выравнивание и переносы
%--------------------------------------
% Избавляемся от переполнений
\sloppy
% Запрещаем разрыв страницы после первой строки абзаца
\clubpenalty=10000
% Запрещаем разрыв страницы после последней строки абзаца
\widowpenalty=10000
%--------------------------------------

%Списки
\usepackage{enumitem}

%Подписи
\usepackage{caption} 

\newenvironment{longlisting}{\captionsetup{type=listing}}{}

%Гиперссылки
\usepackage{hyperref}

\hypersetup{
  colorlinks=true,
  unicode=true,
}

%Рисунки
%--------------------------------------
\DeclareCaptionLabelSeparator*{emdash}{~--- }
\captionsetup[figure]{labelsep=emdash,font=onehalfspacing,position=bottom}
%--------------------------------------

\usepackage{tempora}

%Листинги
%--------------------------------------
\usepackage{minted}

\renewcommand\listingscaption{Листинг}
\setminted[py]{
  frame=single,
  fontsize=\small,
  linenos,
  xleftmargin=1.5em,
}
%--------------------------------------

%%% Математические пакеты %%%
%--------------------------------------
\usepackage{amsthm,amsfonts,amsmath,amssymb,amscd}  % Математические дополнения от AMS
\usepackage{mathtools}                              % Добавляет окружение multlined
\usepackage[perpage]{footmisc}
%--------------------------------------

%--------------------------------------
%			НАЧАЛО ДОКУМЕНТА
%--------------------------------------

\begin{document}

%--------------------------------------
%			ТИТУЛЬНЫЙ ЛИСТ
%--------------------------------------
\begin{titlepage}
\thispagestyle{empty}
\newpage


%Шапка титульного листа
%--------------------------------------
\vspace*{-60pt}
\hspace{-65pt}
\begin{minipage}{0.3\textwidth}
\hspace*{-20pt}\centering
\includegraphics[width=\textwidth]{emblem}
\end{minipage}
\begin{minipage}{0.67\textwidth}\small \textbf{
\vspace*{-0.7ex}
\hspace*{-6pt}\centerline{Министерство науки и высшего образования Российской Федерации}
\vspace*{-0.7ex}
\centerline{Федеральное государственное бюджетное образовательное учреждение }
\vspace*{-0.7ex}
\centerline{высшего образования}
\vspace*{-0.7ex}
\centerline{<<Московский государственный технический университет}
\vspace*{-0.7ex}
\centerline{имени Н.Э. Баумана}
\vspace*{-0.7ex}
\centerline{(национальный исследовательский университет)>>}
\vspace*{-0.7ex}
\centerline{(МГТУ им. Н.Э. Баумана)}}
\end{minipage}
%--------------------------------------

%Полосы
%--------------------------------------
\vspace{-25pt}
\hspace{-35pt}\rule{\textwidth}{2.3pt}

\vspace*{-20.3pt}
\hspace{-35pt}\rule{\textwidth}{0.4pt}
%--------------------------------------

\vspace{1.5ex}
\hspace{-35pt} \noindent \small ФАКУЛЬТЕТ\hspace{80pt} <<Информатика и системы управления>>

\vspace*{-16pt}
\hspace{47pt}\rule{0.83\textwidth}{0.4pt}

\vspace{0.5ex}
\hspace{-35pt} \noindent \small КАФЕДРА\hspace{50pt} <<Теоретическая информатика и компьютерные технологии>>

\vspace*{-16pt}
\hspace{30pt}\rule{0.866\textwidth}{0.4pt}
  
\vspace{11em}

\begin{center}
\Large {\bf Домашняя работа №5} \\ 
\large {\bf по курсу <<Теория искусственных нейронных сетей>>} \\
\large <<Свёрточные нейронные сети>> 

\end{center}\normalsize

\vspace{8em}

\begin{flushright}
  {Студент группы ИУ9-71Б Афанасьев И. \hspace*{15pt}\\ 
  \vspace{2ex}
  Преподаватель Каганов Ю.Т.\hspace*{15pt}}
\end{flushright}

\bigskip

\vfill
 

\begin{center}
\textsl{Москва 2024}
\end{center}
\end{titlepage}
%--------------------------------------
%		КОНЕЦ ТИТУЛЬНОГО ЛИСТА
%--------------------------------------

\renewcommand{\ttdefault}{pcr}

\setlength{\tabcolsep}{3pt}
\newpage
\setcounter{page}{2}

\section{Цель работы}

\begin{enumerate}
  \item Изучение основных архитектур свёрточных нейронных сетей: LeNet-5, VGG-16, ResNet-34.
  \item Обучение свёрточных сетей с использованием различных оптимизаторов и фреймворка PyTorch.
\end{enumerate}

\section{Реализация}

В листинге \ref{lst:hw5.py} приводится исходный код программы на языке Python с использованием фреймворка PyTorch.

\begin{longlisting}
  \caption{Файл \texttt{hw5.py}}
  \inputminted{py}{../sample/hw5.py}
  \label{lst:hw5.py}
\end{longlisting}

\section{Результаты сравнения}

\subsection{Архитектура LeNet-5}

Для обучения нейронной сети с архитектурой LeNet-5 используется датасет MNIST. В таблице \ref{table:lenet-5} приводятся результаты обучения. Точность
измеряется на тестовых данных

\begin{table}[h!]
\centering
\begin{tabular}{||c | c | c | c||}
 \hline
 Оптимизатор & Эпохи & Коэффициент обучения & Точность, $\%$ \\ [0.5ex] 
 \hline\hline
 SGD & 20 & $10^{-2}$ & 98.7 \\ 
 \hline
 AdaDelta & 20 & $10^{-2}$ & 98.8 \\
 \hline
 NAD & 20 & $10^{-2}$ & 99.1 \\
 \hline
 Adam & 20 & $10^{-2}$ & 98.9 \\ [1ex] 
 \hline
\end{tabular}
\caption{Вариация гиперпараметров LeNet-5.}
\label{table:lenet-5}
\end{table}

\subsection{Архитектура VGG-16}

Для обучения нейронной сети с архитектурой VGG-16 используется датасет CIFAR-10. В таблице \ref{table:vgg-16} приводятся результаты обучения.

\begin{table}[h!]
\centering
\begin{tabular}{||c | c | c | c | c||}
 \hline
  Оптимизатор & Эпохи & Коэффициент обучения & Dropout & Точность, $\%$ \\ [0.5ex] 
 \hline\hline
  SGD & 20 & $10^{-2}$ & 0.5 & 73.9 \\ 
 \hline
  AdaDelta & 20 & $10^{-2}$ & 0.5 & 75.9 \\
 \hline
  NAD & 20 & $10^{-2}$ & 0.5 & 80.9 \\
 \hline
  Adam & 20 & $10^{-2}$ & 0.5 & 10.0 \\ [1ex] 
 \hline
\end{tabular}
\caption{Вариация гиперпараметров VGG-16.}
\label{table:vgg-16}
\end{table}

По таблице \ref{table:vgg-16} видно, что с использованием оптимизатора Adam достигается очень низкая точность --- идёт застревание в локальном минимуме.
Изменением гиперпараметров (количество эпох, коэффициент обучения, dropout) решить проблему не удаётся.

\subsection{Архитектура ResNet-34}

Для обучения нейронной сети с архитектурой ResNet-34 используется датасет CIFAR-10. В таблице \ref{table:resnet-34} приводятся результаты обучения.

\begin{table}[h!]
\centering
\begin{tabular}{||c | c | c | c | c||}
 \hline
  Оптимизатор & Эпохи & Коэффициент обучения & Dropout & Точность, $\%$ \\ [0.5ex] 
 \hline\hline
  SGD & 40 & $10^{-2}$ & 0.5 & 65.4 \\ 
 \hline
  AdaDelta & 40 & $10^{-2}$ & 0.5 & 66.3 \\
 \hline
  NAD & 40 & $10^{-2}$ & 0.5 & 75.2 \\
 \hline
  Adam & 20 & $10^{-2}$ & 0.5 & 74.8 \\ [1ex] 
 \hline
\end{tabular}
\caption{Вариация гиперпараметров ResNet-34.}
\label{table:resnet-34}
\end{table}

\end{document}
