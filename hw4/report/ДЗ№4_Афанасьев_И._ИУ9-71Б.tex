\documentclass[a4paper, 14pt]{extarticle}

% Поля
%--------------------------------------
\usepackage{geometry}
\geometry{a4paper,tmargin=2cm,bmargin=2cm,lmargin=3cm,rmargin=1cm}
%--------------------------------------


%Russian-specific packages
%--------------------------------------
\usepackage[T2A]{fontenc}
\usepackage[utf8]{inputenc} 
\usepackage[english, main=russian]{babel}
%--------------------------------------

\usepackage{textcomp}

% Красная строка
%--------------------------------------
\usepackage{indentfirst}               
%--------------------------------------             


%Graphics
%--------------------------------------
\usepackage{graphicx}
\graphicspath{ {./images/} }
\usepackage{wrapfig}
%--------------------------------------

% Полуторный интервал
%--------------------------------------
\linespread{1.3}                    
%--------------------------------------

%Выравнивание и переносы
%--------------------------------------
% Избавляемся от переполнений
\sloppy
% Запрещаем разрыв страницы после первой строки абзаца
\clubpenalty=10000
% Запрещаем разрыв страницы после последней строки абзаца
\widowpenalty=10000
%--------------------------------------

%Списки
\usepackage{enumitem}

%Подписи
\usepackage{caption} 

\newenvironment{longlisting}{\captionsetup{type=listing}}{}

%Гиперссылки
\usepackage{hyperref}

\hypersetup{
  colorlinks=true,
  unicode=true,
}

%Рисунки
%--------------------------------------
\DeclareCaptionLabelSeparator*{emdash}{~--- }
\captionsetup[figure]{labelsep=emdash,font=onehalfspacing,position=bottom}
%--------------------------------------

\usepackage{tempora}

%Листинги
%--------------------------------------
\usepackage{minted}

\renewcommand\listingscaption{Листинг}
\setminted[cpp]{
  frame=single,
  fontsize=\small,
  linenos,
  xleftmargin=1.5em,
}
%--------------------------------------

%%% Математические пакеты %%%
%--------------------------------------
\usepackage{amsthm,amsfonts,amsmath,amssymb,amscd}  % Математические дополнения от AMS
\usepackage{mathtools}                              % Добавляет окружение multlined
\usepackage[perpage]{footmisc}
%--------------------------------------

%--------------------------------------
%			НАЧАЛО ДОКУМЕНТА
%--------------------------------------

\begin{document}

%--------------------------------------
%			ТИТУЛЬНЫЙ ЛИСТ
%--------------------------------------
\begin{titlepage}
\thispagestyle{empty}
\newpage


%Шапка титульного листа
%--------------------------------------
\vspace*{-60pt}
\hspace{-65pt}
\begin{minipage}{0.3\textwidth}
\hspace*{-20pt}\centering
\includegraphics[width=\textwidth]{emblem}
\end{minipage}
\begin{minipage}{0.67\textwidth}\small \textbf{
\vspace*{-0.7ex}
\hspace*{-6pt}\centerline{Министерство науки и высшего образования Российской Федерации}
\vspace*{-0.7ex}
\centerline{Федеральное государственное бюджетное образовательное учреждение }
\vspace*{-0.7ex}
\centerline{высшего образования}
\vspace*{-0.7ex}
\centerline{<<Московский государственный технический университет}
\vspace*{-0.7ex}
\centerline{имени Н.Э. Баумана}
\vspace*{-0.7ex}
\centerline{(национальный исследовательский университет)>>}
\vspace*{-0.7ex}
\centerline{(МГТУ им. Н.Э. Баумана)}}
\end{minipage}
%--------------------------------------

%Полосы
%--------------------------------------
\vspace{-25pt}
\hspace{-35pt}\rule{\textwidth}{2.3pt}

\vspace*{-20.3pt}
\hspace{-35pt}\rule{\textwidth}{0.4pt}
%--------------------------------------

\vspace{1.5ex}
\hspace{-35pt} \noindent \small ФАКУЛЬТЕТ\hspace{80pt} <<Информатика и системы управления>>

\vspace*{-16pt}
\hspace{47pt}\rule{0.83\textwidth}{0.4pt}

\vspace{0.5ex}
\hspace{-35pt} \noindent \small КАФЕДРА\hspace{50pt} <<Теоретическая информатика и компьютерные технологии>>

\vspace*{-16pt}
\hspace{30pt}\rule{0.866\textwidth}{0.4pt}
  
\vspace{11em}

\begin{center}
\Large {\bf Домашняя работа №~4} \\ 
\large {\bf по курсу <<Теория искусственных нейронных сетей>>} \\
\end{center}\normalsize

\vspace{8em}

\begin{flushright}
  {Студент группы ИУ9-71Б Афанасьев И. \hspace*{15pt}\\ 
  \vspace{2ex}
  Преподаватель Каганов Ю.Т.\hspace*{15pt}}
\end{flushright}

\bigskip

\vfill
 

\begin{center}
\textsl{Москва 2024}
\end{center}
\end{titlepage}
%--------------------------------------
%		КОНЕЦ ТИТУЛЬНОГО ЛИСТА
%--------------------------------------

\renewcommand{\ttdefault}{pcr}

\setlength{\tabcolsep}{3pt}
\newpage
\setcounter{page}{2}

\section{Цель работы}

\begin{enumerate}
  \item Сравнительный анализ современных методов оптимизации (SGD, NAG, Adagrad, ADAM) на примере многослойного персептрона.
  \item Использование генетического алгоритма для оптимизации гиперпараметров многослойного персептрона.
\end{enumerate}

\section{Реализация}

В реализации используется каркас многослойного персептрона, разработанный в домашнем задании №2. Программа написана на языке C++.

В листингах \ref{lst:activation_function.h}, \ref{lst:cost_function.h} приводится исходный код функций активации и стоимости.
В листингах \ref{lst:data_supplier.h}, \ref{lst:data_supplier.cc} приводится исходный код загрузки датасета MNIST.
В листингах \ref{lst:perceptron.h} и \ref{lst:perceptron.cc} приводится исходный код каркаса многослойного персептрона и 
методов оптимизации: SGD, NAG, Adagrad, Adam.

В листингах \ref{lst:chromosome.h} и \ref{lst:chromosome.cc} приводятся классы хромосомы, используемые генетическим алгоритмом.
В листингах \ref{lst:fitness_function.h} и \ref{lst:fitness_function.cc} приводятся классы функции приспособленности. В качестве
значения приспособленности используются обратные значения функций стоимости на тестовых данных.
В листингах \ref{lst:genetic_algorithm.h} и \ref{lst:genetic_algorithm.cc} приводится реализация генетического алгоритма. Отбор производится
по методу рулетки. При скрещивании потомки в разных пропорциях получают родительские характеристики, а при мутации случайным
образом изменяется некоторый ген хромосомы (в частности, алгоритм параметризуется долями хромосом, подвергающихся скрещиванию и мутации).

В листинге \ref{lst:main.cc} приводится исходный код \texttt{main}-файла программы.

\begin{longlisting}
  \caption{Файл \texttt{activation\_function.h}}
  \inputminted{cpp}{../src/activation_function.h}
  \label{lst:activation_function.h}
\end{longlisting}

\begin{longlisting}
  \caption{Файл \texttt{cost\_function.h}}
  \inputminted{cpp}{../src/cost_function.h}
  \label{lst:cost_function.h}
\end{longlisting}

\begin{longlisting}
  \caption{Файл \texttt{data\_supplier.h}}
  \inputminted{cpp}{../src/data_supplier.h}
  \label{lst:data_supplier.h}
\end{longlisting}

\begin{longlisting}
  \caption{Файл \texttt{data\_supplier.cc}}
  \inputminted{cpp}{../src/data_supplier.cc}
  \label{lst:data_supplier.cc}
\end{longlisting}

\begin{longlisting}
  \caption{Файл \texttt{perceptron.h}}
  \inputminted{cpp}{../src/perceptron.h}
  \label{lst:perceptron.h}
\end{longlisting}

\begin{longlisting}
  \caption{Файл \texttt{perceptron.cc}}
  \inputminted{cpp}{../src/perceptron.cc}
  \label{lst:perceptron.cc}
\end{longlisting}

\begin{longlisting}
  \caption{Файл \texttt{chromosome.h}}
  \inputminted{cpp}{../src/chromosome.h}
  \label{lst:chromosome.h}
\end{longlisting}

\begin{longlisting}
  \caption{Файл \texttt{chromosome.cc}}
  \inputminted{cpp}{../src/chromosome.cc}
  \label{lst:chromosome.cc}
\end{longlisting}

\begin{longlisting}
  \caption{Файл \texttt{fitness\_function.h}}
  \inputminted{cpp}{../src/fitness_function.h}
  \label{lst:fitness_function.h}
\end{longlisting}

\begin{longlisting}
  \caption{Файл \texttt{fitness\_function.cc}}
  \inputminted{cpp}{../src/fitness_function.cc}
  \label{lst:fitness_function.cc}
\end{longlisting}

\begin{longlisting}
  \caption{Файл \texttt{genetic\_algorithm.h}}
  \inputminted{cpp}{../src/genetic_algorithm.h}
  \label{lst:genetic_algorithm.h}
\end{longlisting}

\begin{longlisting}
  \caption{Файл \texttt{genetic\_algorithm.cc}}
  \inputminted{cpp}{../src/genetic_algorithm.cc}
  \label{lst:genetic_algorithm.cc}
\end{longlisting}
 
\begin{longlisting}
  \caption{Файл \texttt{main.cc}}
  \inputminted{cpp}{../src/main.cc}
  \label{lst:main.cc}
\end{longlisting}

\section{Результаты сравнения}

Оптимальные гиперпараметры для обучения многослойного персептрона с различными оптимизаторами определялись с помощью генетического алгоритма.
Фиксированными были количество эпох обучения (100) и размер пакета данных (100). Коэффициент обучения определялся в промежутке $[0.001, 1]$,
количество скрытых слоёв --- от 0 до 4, количество нейронов в скрытых слоях --- от 10 до 40. Параметры генетического алгоритма:
\begin{itemize}
  \item количество популяций: 10;
  \item размер популяции: 60;
  \item доля особей, подвергающихся скрещиванию: 40\%;
  \item доля особей, подвергающихся мутации: 15\%.
\end{itemize}

Функции активации скрытых слоёв персептрона --- Leaky ReLU, выходного слоя --- Softmax. Функция стоимости --- перекрёстная энтропия.

\subsection{SGD}

Для SGD-оптимизатора определены следующие оптимальные гиперпараметры:
\begin{itemize}
  \item коэффициент обучения: 0.0959074;
  \item количество скрытых слоёв: 1;
  \item количество нейронов в скрытых слоях: 28.
\end{itemize}
Точность работы пересептрона с указанными гиперпараметрами на тренировочных данных составляет $97.06\%$, на тестовых данных --- $95.80\%$.

\subsection{NAG}

Для NAG-оптимизатора с коэффициентом $\gamma = 0.9$ определены следующие оптимальные гиперпараметры:
\begin{itemize}
  \item коэффициент обучения: 0.0100863;
  \item количество скрытых слоёв: 1;
  \item количество нейронов в скрытых слоях: 28.
\end{itemize}
Точность работы пересептрона с указанными гиперпараметрами на тренировочных данных составляет $97.12\%$, на тестовых данных --- $95.65\%$.

\subsection{Adagrad}

Для Adagrad-оптимизатора с $\varepsilon = 10^{-8}$ определены следующие оптимальные гиперпараметры:
\begin{itemize}
  \item коэффициент обучения: 0.936544;
  \item количество скрытых слоёв: 1;
  \item количество нейронов в скрытых слоях: 36.
\end{itemize}
Точность работы пересептрона с указанными гиперпараметрами на тренировочных данных составляет $94.72\%$, на тестовых данных --- $93.95\%$.

\subsection{Adam}

Для Adam-оптимизатора с $\varepsilon = 10^{-8}, \beta_1 = 0.9, \beta_2 = 0.999$ определены следующие оптимальные гиперпараметры:
\begin{itemize}
  \item коэффициент обучения: 0.0462101;
  \item количество скрытых слоёв: 3;
  \item количество нейронов в скрытых слоях: 35.
\end{itemize}
Точность работы пересептрона с указанными гиперпараметрами на тренировочных данных составляет $97.38\%$, на тестовых данных --- $96.20\%$.

\section{Вывод}

Наилучший результат в $96.20\%$ точности на тестовых данных был достигнут с оптимизатором Adam, который сочетает в себе идеи оптимизаторов
NAG и Adagrad. Благодаря использованию генетического алгоритма удалось автоматически определить оптимальные гиперпараметры для обучения
персептрона с различными оптимизаторами.

\end{document}
